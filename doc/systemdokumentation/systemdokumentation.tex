\documentclass{TDP003mall}
\usepackage{listings}
\usepackage{color}
\usepackage{listings}
\definecolor{grey}{gray}{0.9}


\lstset{%
language=Lisp,
basicstyle=\small,
backgroundcolor=\color{grey},
mathescape=true}

\newcommand{\version}{Version 0.5}
\author{IP1 2015}
\title{Installationsmanual \\för Portföljsystem}
\date{2015-09-21}
\rhead{IP1 2015}



\begin{document}
\projectpage
\section{Revisionshistorik}
\begin{table}[!h]
\begin{tabularx}{\linewidth}{|l|X|l|}\hline
Ver. & Revisionsbeskrivning & Datum \\\hline
0.5 & Korrekturläst och åtgärdat språkliga fel & 150925\\\hline
0.4 & Automatisk och manuell installation fungerar & 150923\\\hline
0.3 & Bugtestat manualen & 150923 \\\hline
0.2 & Template med grunduppgifter skapad & 150921 \\\hline
0.1 & Skapat Flask-avsnittet  & 150921 \\\hline




\end{tabularx}
\end{table}

\section{Introduktion/Förutsättningar}

I denna installationsmanual kommer ni få reda på hur man laddar ner, installerar och använder vårt portfoliosystem.\\\\Denna installationsmanual förutsätter att ni redan har:

\begin{itemize}
  \item En dator med Linux Mint 17 "Qiana" installerat
  \item Tillgång till en root-användare på den nämnda datorn
  \item En internetuppkoppling
\end{itemize}Det är rekommenderat att ni har lite erfarenhet inom Bash då det kommer att förenkla installationen samt användandet av portföljen
Om ni inte har det så kan ni läsa kapitel två på nedanstående länk, som är en guide över de viktigaste kommandona: 
\url{http://www.tldp.org/LDP/intro-linux/html/sect_02_01.html}

\section{Installera Flask (och Jinja)}

Öppna upp terminalen, förslagsvis via startmenyn.\\
I sektionen nedan går vi genom både hur man genomför en automatisk installation, eller en manuell installation. Text med grå bakgrund är kommandon som skall skrivas in i terminalen. 
\subsection{Automatisk installation}

\begin{enumerate}
  \item Då vi i resten av guiden förutsätter att vi arbetar i ett fullt uppdaterat system, kör följande kommando.
      \begin{lstlisting}
  sudo apt-get update  
      \end{lstlisting}
 \item För att kunna ladda ner filer direkt från terminalen: Installera ``wget''. 
    \begin{lstlisting}
  sudo apt-get install wget
    \end{lstlisting}
  \item Ladda nu ned installationsfilen för Flask med ``wget''.
      \begin{lstlisting}
  wget http://www-und.ida.liu.se/~johbo462/TDP003/flaskinstall/install.sh
      \end{lstlisting}
  \item Kör sedan filen för att påbörja den automatiska installationen av Flask.
      \begin{lstlisting}
  sudo sh install.sh
      \end{lstlisting}
Din terminal fungerar nu som en virtuell server.\pagebreak

\end{enumerate}

\subsection{Manuell installation}

\begin{enumerate}
  \item Se till att du startar från din hemkatalog.
          \begin{lstlisting}
  cd  
      \end{lstlisting}
  \item Då vi i resten av guiden förutsätter att vi arbetar i ett fullt uppdaterat system, kör följande kommando.
      \begin{lstlisting}
  sudo apt-get update  
      \end{lstlisting}
  \item För att kunna ladda ner filer direkt från terminalen: Installera ``wget''. 
    \begin{lstlisting}
  sudo apt-get install wget
    \end{lstlisting}
  \item Skapa en mapp som heter flaskinstall där installationsfilen kommer hamna.
    \begin{lstlisting}
  mkdir flaskinstall
    \end{lstlisting}
  \item Öppna nu mappen ``flaskinstall'' du precis skapade.
    \begin{lstlisting}
  cd flaskinstall
    \end{lstlisting}
  \item Installera nu inställningsverktyg för Python. De behövs för resten av stegen.
    \begin{lstlisting}
  sudo apt-get install python-setuptools
    \end{lstlisting}
  \item Ladda ner filen get-pip.py med ``wget''.
    \begin{lstlisting}
  wget https://raw.github.com/pypa/pip/master/contrib/get-pip.py
    \end{lstlisting}
  \item Öppna pythonfilen som installerar pip.
    \begin{lstlisting}
  sudo python3 get-pip.py
    \end{lstlisting}
  \item Installera senaste pip-versionen.
    \begin{lstlisting}
  sudo pip install -U pip
    \end{lstlisting}
  \item Installera Flask där Jinja2 ingår. 
    \begin{lstlisting}
  sudo pip3.4 install flask
    \end{lstlisting}
  \item Ladda ner test-filen för Flask med ``wget''. 
    \begin{lstlisting}
  wget http://www-und.ida.liu.se/~johbo462/TDP003/flaskinstall/test-flask.py
    \end{lstlisting}
  \item Kör test-filen för att starta servern. 
    \begin{lstlisting}
  python3 test-flask.py
    \end{lstlisting}

Din terminal fungerar nu som en virtuell server. 
\end{enumerate}

\section{Testa om installationen lyckades}

\begin{enumerate}
  \item Öppna din webbläsare.
  \item Skriv in följande i adressfältetet: 127.0.0.1:5000

Om det står ``flask hälsar dig välkommen'' i din webbläsare har allt fungerat som det ska.\\
Grattis!
% det mesta är täckt i de andra commitsen, men en fin kommentar skadar inte!
% Rättstavade jättemycket. Vi är bäst. Bra jobbat allihopa, dasawrap!
\end{enumerate}


\end{document}
